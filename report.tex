\documentclass{article}
\usepackage[utf8]{inputenc}
\usepackage{alphabeta}

\usepackage{sectsty}
\sectionfont{\centering}

\title{Εργασία Τεχνητής Νοημοσύνης 2020}
\author{Σπύρος Μαντέλος Α.Μ: 4104\\Χρήστος Γεωργίου Μουσσές Α.Μ: 4206\\Γεώργιος Μήτσης Α.Μ: \textit{unknown}}
\date{}

\begin{document}
\maketitle
\section*{Άσκηση 1}
Στην πρώτη άσκηση υλοποιήθηκαν οι αλγόριθμοι \textbf{UCS} και \textbf{A*} στα πλαίσια του προβλήματος μετακίνησης σφαιρών με το μικρότερο δυνατό κόστος.
\newpage
\section*{Άσκηση 2}
Στη δεύτερη άσκηση υλοποιήθηκε ο αλγόριθμος \textbf{MINIMAX} στα πλαίσια του παιγνίου δύο παικτών SOS.
\subsection*{Αλγόριθμος}
Η επιλογή της ευρετικής συνάρτησης αξίας έγινε με βάση τον κανόνα τερματισμού του παιγνίου λαμβάνοντας υπόψη επιπρόσθετα και το βάθος της απόφασης στο δέντρο απόφασης του αλγόριθμου.\\\\
Πιο συγκεκριμένα ορίστηκαν οι εξής τιμές για την αξία των τελικών καταστάσεων:
\begin{itemize}
    \item \textit{(10 – \textbf{depth})} αν ο παίκτης υπολογιστής εκτελεί νικητήρια κίνηση σε βάθος απόφασης \textit{\textbf{depth}}.
    \item \textit{(-10 + \textbf{depth})} αν ο παίκτης χρήστης εκτελεί νικητήρια κίνηση σε βάθος απόφασης \textit{\textbf{depth}}.
    \item \textit{0} αν το παίγνιο λήγει σε ισοπαλία (κανένας παίκτης δεν έχει νικήσει και το ταμπλό είναι γεμάτο).
\end{itemize}
Κάνοντας αυτή την αξιολόγηση, η ευρετική συνάρτηση αξίας μπορεί να δώσει στις τερματικές καταστάσεις ως τιμές τους ακέραιους στο διάστημα
[-9, -2] για νικητήρια κίνηση του παίκτη χρήστη, 0 για ισοπαλία και τους ακέραιους στο διάστημα [2, 9] για
νικητήρια κίνηση του παίκτη υπολογιστή. Εδώ χρησιμοποιήθηκε το γεγονός πως το μέγιστο βάθος του δέντρου
απόφασης είναι 8 (στην αρχή του παιχνιδιού υπάρχουν 8 διαθέσιμες θέσεις
για τοποθέτηση είτε S είτε O).\\\\
Με αυτόν τον τρόπο δίνεται προτεραιότητα στην επιλογή κινήσεων που οδηγούν τον παίκτη υπολογιστή σε νίκες που βρίσκονται σε μικρό βάθος απόφασης, “τιμωρόντας” παράλληλα την επιλογή
κινήσεων που οδηγούν τον παίκτη υπολογιστή σε ήττες που βρίσκονται σε μικρό βάθος απόφασης.

\subsection*{Υλοποίηση}
Η υλοποίηση του αλγορίθμου έγινε σε γλώσσα \textbf{Java} και έγινε με τη χρήση αναδρομής όπως αναφέρεται και στην εκφώνηση.\\\\
Αρχικά, υλοποιήθηκε το
παιχνίδι μεταξύ δύο χρηστών (με επιλογή ονομάτων), και αφότου βεβαιώθηκε η σωστή λειτουργία του, έπειτα
υλοποιήθηκε ο αλγόριθμος \textbf{MINIMAX} (επιλογή ονόματος \textbf{minimax}). Για την επιβεβαίωση της σωστής λειτουργίας της υλοποίησης του αλγόριθμου, μεταξύ άλλων συγκρίθηκε ο αριθμός όλων των πιθανών καταστάσεων του παιγνίου στο πλήρες δέντρο καταστάσεων, με τον πραγματικό αριθμό καταστάσεων που επισκέπτεται ο αλγόριθμος κατά την υλοποίησή του στο δέντρο απόφασης. Πιο συγκεκριμένα, υπολογίστηκε ο αριθμός των κόμβων του πλήρους δένδρου καταστάσεων (πλην της ρίζας) σε κάποια διαμόρφωση του ταμπλό, ο οποίος προκύπτει από το άθροισμα: $$\sum_{i=1}^{α}\left[\left(\prod_{n=α+1-i}^{α}n\right)2^{i}\right]$$ όπου το α είναι ο αριθμός των κενών κελιών σε κάποια διαμόρφωση του ταμπλό.\\\\
Για παράδειγμα, όταν επιλέγεται η πρώτη κίνηση (όπου υπάρχουν 8 κενά κελιά στη συγκεκριμένη διαμόρφωση), το πλήρες δέντρο καταστάσεων περιλαμβάνει, με βάση τον παραπάνω τύπο, 17.017.968 κόμβους (πλην της ρίζας). Η υλοποίηση του αλγόριθμου επισκέπτεται 10.449.792 κόμβους (πλην της ρίζας), κάτι το οποίο είναι αναμενόμενο, αφού υπάρχουν και καταστάσεις σε βάθος μικρότερο από 8, όπου το παίγνιο λήγει με νίκη κάποιου παίκτη ο οποίος έχει σχηματίσει αλυσίδα \textbf{SOS}, χωρίς να έχει συμπληρωθεί πλήρως το ταμπλό του παίγνιου. Έτσι τελικά, ο αριθμός των κόμβων που ελέγχει ο αλγόριθμος στο δέντρο απόφασης, είναι πάντα μικρότερος, ή ίσος από τον αριθμό των κόμβων του πλήρους δέντρου καταστάσεων. \\\\
Γίνεται άμεσα αντιληπτό πως επειδή ο όγκος των διαμορφώσεων που καλείται να ελέγξει ο παίκτης υπολογιστής είναι πολύ μεγάλος στις αρχικές κινήσεις, ο χρόνος που απαιτείται για την επεξεργασία τους είναι σαφώς μεγαλύτερος από το χρόνο που χρειάζεται  σε μετέπειτα κινήσεις. Δίνεται παρακάτω παράδειγμα ενός ενδεικτικού παιχνιδιού σε γύρους όπου ο υπολογιστής παίζει πρώτος με τον αριθμό διαμορφώσεων που ελέγθηκαν σε κάθε γύρο.
\begin{enumerate}\addtocounter{enumi}{-1}
	\centering
    \item Αρχική κατάσταση: 2 1 O
    \item Κίνηση παίκτη \textbf{minimax}: 1 1 O\\ Διαμορφώσεις που ελέγθηκαν: 10.449.792 
    \item Κίνηση παίκτη χρήστη: 2 2 S
    \item Κίνηση παίκτη \textbf{minimax}: 1 2 S\\ Διαμορφώσεις που ελέγθηκαν: 69.228
    \item Κίνηση παίκτη χρήστη: 3 2 S
    \item Κίνηση παίκτη \textbf{minimax}: 1 3 S\\ Διαμορφώσεις που ελέγθηκαν: 620 
    \item Κίνηση παίκτη χρήστη: 3 3 O
    \item Κίνηση παίκτη \textbf{minimax}: 2 3 S\\ Διαμορφώσεις που ελέγθηκαν: 12
    \item Κίνηση παίκτη χρήστη: 3 1 O\\Λήξη του παίγνιου σε ισοπαλία
\end{enumerate}
Αν και στην άσκηση ζητείται η υλοποίηση της περίπτωσης που ο παίκτης \textbf{MAX} (παίκτης υπολογιστής) παίζει 
πρώτος, η υλοποίηση του αλγόριθμου επιτρέπει στον παίκτη υπολογιστή να παίξει επίσης δεύτερος. Επιπρόσθετα,
υλοποιήθηκε αλγόριθμος που επιτρέπει στον παίκτη υπολογιστή να παίζει, επιλέγοντας τυχαία τις συντεταγμένες και το σύμβολο σε κάθε γύρο (επιλογή ονόματος \textbf{cpu}).
\end{document}
